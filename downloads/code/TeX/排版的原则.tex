%!TEX TS-program = xelatex
%!TEX encoding = UTF-8 Unicode
\documentclass[12pt,a4paper]{article}
\XeTeXlinebreaklocale "zh"     					% 针对中文进行断行
\XeTeXlinebreakskip = 0pt plus 1pt minus 0.1pt  % 给予TeX断行一定自由度
\linespread{1.5}                                % 1.5倍行距
\usepackage{fontspec,xunicode,}
%\usepackage{xltxtra}
\usepackage[small, sf]{titlesec}

\newfontfamily\youyuan{YouYuan} 
\newfontfamily\hwcaiyun{STCaiyun} 
\newfontfamily\hwhupo{STHupo} 
\newfontfamily\yaoti{FZYaoTi} 
\newfontfamily\kaiti{KaiTi} 
\newfontfamily\xsong{NSimSun} 
\newfontfamily\hwsong{STSong} 
\newfontfamily\yahei{Microsoft YaHei} 
\newfontfamily\fangsong{FangSong} 
\newfontfamily\song{SimSun} 
\newfontfamily\hwfangsong{STFangsong} 
\newfontfamily\weiti{STXinwei} 
\newfontfamily\heiti{SimHei} 
\newfontfamily\hwxingkai{STXingkai} 
\newfontfamily\hwlishu{STLiti} 
\newfontfamily\zhongsong{STZhongsong} 
\newfontfamily\shuti{FZShuTi} 
\newfontfamily\hwhei{STXihei} 
\newfontfamily\lishu{LiSu} 
\newfontfamily\hwkai{STKaiti}

\setmainfont[BoldFont={SimHei}]{SimSun}
\setsansfont[Mapping=tex-text]{Microsoft Sans Serif} 
\setmonofont{Courier New}


\titleformat*{\section}{\youyuan}
\titleformat*{\subsection}{\youyuan}


\begin{document}
\begin{center}
{\Large \textbf {排版的原则}}\\[5mm] %指令末端之 \\ 為換行指令 , 指示本行結束 ; 緊接著的 [5mm] 則設定行距加大 0.5 公分。 
吴聪敏\\[2mm]
2005.03
\end{center}

\bigskip %定行内距,medskip-half,small-qua
\fontsize{12}{20pt}\selectfont %字体 12pt,行距20pt
排版的目的是为了有效率地表达你的意见,
让读者容易阅读或吸收。

\section{常见的排版错误}
桌上排版系统普及之后,每个人都能排版,
但这不表示每个人都能做出好的排版。
关于字体之使用,常见的错误如下:
\begin{itemize} %无序,不带标号
\item 楷体与仿宋体等可用于标题,
  但不宜用于正文;正文请使用明体。
\item 较长之英文段落应使用纯英文字体编排,
  如 Times New Roman,Book Antiquarian 等。
  使用中文之细明体或楷体之英文字体,
  结果保证惨不忍睹。
\item 欲强调之字词,请使用粗黑体或圆体,勿加底线。
\end{itemize}
Windows 系统提供明体与楷体两种字体,
许多人只能使用楷体排版正文或投影片,
但排版效果奇差无比。

除了字体之外,版面应注意之细节如下:
\begin{enumerate}\itemsep=-2pt %带编号,設定使項目之間距比內定值小 2pt
\item 文章前端请写下题目,作者名字,日期。
\item 版面之一行勿拉太长,行距不应过小。
\item 标点符号不应出现在一行前端(避头点)。
\item 标示注解之号码在句子标点符号之后。
\item 章节编号使用阿拉伯数字,如1.1节,2.3节;
避免使用“一、二节”,或“二、三节”。
\item 阿拉伯数字“1,534”比“一千五百三十四”清楚。
\end{enumerate}

\section{图形与表格}
图形与表格是表现咨询的有效方法,
但是图表作的不好,无法达到目标。
常见的表格排版错误是格线太多,
但另一个问题是单位不清楚。
大型图表应移于版面的上方或下方,而非排版于段落中间。
非万不得已,表格勿拆为两页。

\par\vfill\jobname.tex %par新起段落 vfill將段落之間距儘可能拉大
\end{document}

























